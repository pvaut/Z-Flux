\chapter{Some tricks}

\section{Co-moving the camera with a suframe}





One way of achieving this is by inverting the total transformation of the subframe, and applying it to the total scene.
For example, suppose
\begin{itemize}
\item root.SC is the scene
\item root.SC.Universe.SolarSystem.Earth.Inclin.Globe is the object the camera should track
\item root.viewports.main is the viewport that needs to be adjusted
\end{itemize}

In the render loop, do:\\
\sourcecode{
root.SC.Universe.transf.reset;\\
tf1=root.SC.Universe.SolarSystem.Earth.Inclin.Globe.obj.totaltransformation;\\
tf2=tf1;tf2.invert;\\
root.viewports.main.transf=tf2;//apply the inverse transformation to the viewport-> object becomes positioned in the origin\\
}


ALTERNATIVE:
for added accuracy (e.g. in case of zooming in on small objects in large scenes, such as the earth in the solar system),
you can apply the inverse transformation on the deepest subframe:

\sourcecode{
root.sc.universe.transf=tf2;\\
root.sc.light0pos=tf2*point(0,0,0);\\
}
